
\section{Analisi dei requisiti}
\subsection{Descrizione testuale}
Si vuole realizzare una base di dati per l'applicazione \textit{Gestionale servizi del diporto} che dovrà gestire tutti i dati relativi ad un marina e le \textbf{imbarcazioni} che vi sosteranno, i \textbf{servizi} offerti e i dati relativi ai \textbf{clienti}.

Le fuznzioni desiderate prevedono: il calcolo delle \textbf{fatture} dei clienti; la gestione di \textbf{soste} e \textbf{prenotazioni} con recupero dei posti disponibili; la presentazione dei \textbf{servizi} aperti ai clienti.

Il marina in esame ha circa 250 \textbf{posti barca}, numero che cambia solo in caso di ristrutturazioni particolari o durante ampliamenti del marina. Il traffico di \textbf{imbarcazioni}, persone ed informazioni è quindi commisurato a questo numero di \textbf{posti barca}.

Alcuni dei \textbf{clienti} del marina sono \textit{occasionali}, ad esempio persone che effettuano crociere e navigano di porto in porto; Altri clienti sono \textit{abituali} e tengono ormeggiate le loro imbarcazioni nel marina per molti mesi all'anno se non addirittura a tempo indeterminato. Dei primi si vuole sapere quante volte hanno ormeggiato nel marina, dei secondi si vuole tenere nota del loro sconto personale applicato alle fatture. Spesso un \textit{cliente occasionale} diventa un \textit{cliente abituale} e quando questo accade gli viene assegnato uno sconto deciso per persona.

Dei \textbf{moli}\footnote{Sinonimo di posto barca} si vuole tenere traccia della loro occupazione nel tempo per fini statistici\footnote{Ad esempio il molo preferito da una determinata persona o il molo più utilizzato}. Per ogni \textbf{molo}, identificato da un numero positivo, è importante memorizzare le dimensioni(Molto importanti, poiché solo le imbarcazioni più piccole di queste dimensioni potranno ormeggiare in questo molo),il prezzo giornaliero , gli allacciamenti\footnote{Si intendono allacciamenti idrici, elettrici, di aria compressa} e lo stato di occupazione.

Gli \textbf{allacciamenti} sono identificati dal nome e hanno un'unità di misura ed un costo unitario.

Relativamente agli \textbf{allacciamenti} ne vengono anche registrati i \textbf{consumi} da parte di ogni cliente, come procedura interna del marina viene fatta una lettura del contatore ogni inizio del mese\footnote{Chiaramente solo se c'è una barca in sosta}. Se l'imbarcazione arriva e/o parte in momenti differenti dall'inizio del mese la lettura viene fatta in quelle due occasioni.

Questi \textbf{consumi}, assieme alle \textbf{soste} effettuate, vengono poi fatturati al cliente e viene registrato se la \textbf{fattura} è stata pagata oppure no.
La \textbf{fattura} è identificata univocamente dal cliente destinatario e la data di scadenza della stessa.

Da quando l'applicativo verrà messo in funzione si inizieranno a memorizzare le \textbf{soste}.

Le \textbf{imbarcazioni} che sosteranno nel marina sosteranno nel tempo in più moli differenti, e chiaramente i \textbf{moli} accoglieranno nel tempo imbarcazioni differenti, ma non contemporaneamente. Un \textbf{molo} può infatti accogliere solo un' \textbf{imbarcazione} alla volta.

Di ogni \textbf{imbarcazione} vengono registrati codice internazionale univoco, la bandiera battuta, il nome del capitano, le dimensioni\footnote{Definite come: Pescaggio,Larghezza e lunghezza fuoritutto con acronimo inglese LOA}, il numero di posti letto e se è presente anche il nome. Queste informazioni vengono memorizzate assieme ai dati dei \textbf{clienti}, ovvero i proprietari delle suddette \textbf{imbarcazioni} per onorare gli obblighi di registrazione doganale.

I \textbf{clienti} sono coloro che possiedono un'\textbf{imbarcazione} ormeggiata all'interno del marina e di loro ne viene salvato il nome, il cognome, il codice fiscale, gli eventuali contatti, la residenza, la cittadinanza, la data di nascita.

Ogni \textbf{cliente} può quindi prenotare un molo del marina preventivamente. Ogni \textbf{prenotazione} è caratterizzata da una data prevista di arrivo, una data prevista di partenza, il molo interessato, ed il cliente prenotante.

Per controllare se un \textbf{molo} è disponibile bisogna quindi controllare se il \textbf{molo} è al momento occupato e inoltre se esistono già delle \textbf{prenotazioni} per le date in cui si prevede che l'\textbf{imbarcazione} sosti. Se la \textbf{sosta} prevista è a tempo indeterminato ma una prenotazione parte più tardi,l'imbarcazione può essere spostata a sostare in un altro molo non prenotato.

Il marina offre poi una serie di \textbf{servizi} utili ai naviganti\footnote{Ad esempio lavanderie a gettoni, ristoranti, cantieri} . Ogni \textbf{servizio} è caratterizzato da un nome unico e gli \textbf{orari} in cui è aperto nei vari giorni di una settimana.

Ogni servizio è amministrato da un \textbf{addetto} del quale ci interessa la data di inizio contratto e la fine. Se la fine non è segnata allora lavorerà li a tempo indeterminato. Ogni \textbf{addetto} è definito univocamente dalla data di inizio contratto e il servizio che gestisce.

\subsection{Glossario dei termini}

\begin{center}
    \begin{tabularx}{\textwidth}{|>{\RaggedRight}p{2cm}|p{7cm}|p{2.4cm}|>{\RaggedRight}X|}
        \hline
        \rowcolor{gray!30}
        \textbf{Entità} & \textbf{Descrizione} & \textbf{Sinonimi} & \textbf{collegamenti} \\
        \hline
        Imbarcazione & Il mezzo di trasporto ormeggiato nei moli &  & Cliente, Sosta\\
        
        \hline
        Molo & Il luogo in cui le imbarcazioni sostano & Posto barca, ormeggio & Sosta, Prenotazione, Allacciamento\\
        
        \hline
        Cliente& Persona che possiede un'imbarcazione ormeggiata nel  marina & Armatore & Entità figlia di Persona, Imbarcazione, Consumo, Prenotazione\\
        
        \hline
        Cliente occasionale &Persone che viaggiano salturiamente & & Entità figlia di Cliente  \\
         
        \hline
        Cliente abituale & Persone che utilizzano stabilmente un marina &  & Entità figlia di Cliente \\
          
        \hline
        Persona& Tutte le persone che frequentano un marina   &  & Padre di Cliente e Addetto \\  
        
        \hline
        Prenotazione & Tutte le prenotazioni dei clienti &  & Molo, cliente \\
        
        \hline
        Servizio & Attività ti varia utilità rivolte ai clienti del marina &  & Periodo apertura, addetto \\
        
        \hline
        Addetto & Persona che gestisce un servizio & Responsabile & Servizio, Persona \\
         
        \hline
        Sosta & Possibilità data ai clienti delle imbarcazioni di un marina  & & Imbarcazione, Molo \\
        
        \hline
        Allacciamento & Un servizio consumabile tipo acqua o elettricità disponibile ad un molo & Fornitura & Molo, Consumo\\
        
        \hline
        Orario di apertura & Orari in cui un servizio è aperto durante la settimana & & Servizio\\
       
        \hline
        Fattura & Ricevuta sui consumi e le soste effettuate da parte dei clienti & ricevuta, scontrino & Cliente, Consumo, Sosta\\
      
        \hline
        Consumo & Utilizzo misurato di un allacciamento presente ai moli & & Allacciamento, Cliente, Fattura \\
        
        \hline
    \end{tabularx}
\end{center}

\subsection{Strutturazione dei requisiti}

\begin{center}
    \begin{tabularx}{\textwidth}{|X|}
        \hline
        \rowcolor{gray!30}
        \multicolumn{1}{|c|}{\textbf{Frasi relative a Marina}}\\
        \hline
        Ogni marina solitamente contiene da 50 fino ad anche 500posti barca. \\
        
        Ognuno di questi marina è caratterizzato da un indirizzo, \textit{un nome}, e \textit{le coordinate geografiche} a cui i naviganti possono dirigere per raggiungerlo. Esso possiede inoltre una quantità più o meno varia di moli\\
        
        Nel marina sosteranno le imbarcazioni,queste imbarcazioni sosteranno nel tempo in più moli differenti.\\
        \hline
    \end{tabularx}
\end{center}

\begin{center}
    \begin{tabularx}{\textwidth}{|X|}
        \hline
        \rowcolor{gray!30}
        \multicolumn{1}{|c|}{\textbf{Frasi relative a Imbarcazione}}\\
        \hline
        queste imbarcazioni sosteranno nel tempo in più moli differenti.\\
        Di ogni \textbf{imbarcazione} vengono registrati codice internazionale, la bandiera battuta,l'armatore, il capitano le dimensioni, il numero di posti letto e se è presente anche il nome. Queste informazioni vengono memorizzate assieme ai dati dei clienti per onorare gli obblighi di registrazione portuale.\\
        \hline
    \end{tabularx}
\end{center}

\begin{center}
    \begin{tabularx}{\textwidth}{|X|}
        \hline
        \rowcolor{gray!30}
        \multicolumn{1}{|c|}{\textbf{Frasi relative a Molo}}\\
        \hline
        moli di cui si vuole tenere traccia l’occupazione nel tempo per fini statistici. per ogni molo sono registrate le dimensioni(Molto importanti, poiché solo le imbarcazioni più piccole di queste dimensioni potranno ormeggiare in questo molo),gli allacciamenti e lo stato di occupazione.\\
        \hline
    \end{tabularx}
\end{center}

\begin{center}
    \begin{tabularx}{\textwidth}{|X|}
        \hline
        \rowcolor{gray!30}
        \multicolumn{1}{|c|}{\textbf{Frasi relative a Cliente}}\\
        \hline
        Alcuni dei clienti del marina sono occasionali, ad esempio persone che effettuano crociere e bazzicano di porto in porto; Altri clienti sono abituali e tengono ormeggiate le loro imbarcazioni nel marina per molti mesi all'anno se non addirittura a tempo indeterminato.\\
        
        I clienti sono coloro che ormeggiano un'imbarcazione all'interno del marina e di loro ne viene salvato il nome, il cognome, il codice fiscale, gli eventuali contatti, la residenza, la cittadinanza, la data di nascita.\\

Ogni cliente può quindi prenotare un molo del marina preventivamente.\\
        \hline
    \end{tabularx}
\end{center}

\begin{center}
    \begin{tabularx}{\textwidth}{|X|}
        \hline
        \rowcolor{gray!30}
        \multicolumn{1}{|c|}{\textbf{Frasi relative a Prenotazione}}\\
        \hline
        Ogni \textbf{prenotazione} è caratterizzata da una data prevista di arrivo, una data prevista di partenza, l'imbarcazione interessata, ed il cliente prenotante.\\
        \hline
    \end{tabularx}
\end{center}

\begin{center}
    \begin{tabularx}{\textwidth}{|X|}
        \hline
        \rowcolor{gray!30}
        \multicolumn{1}{|c|}{\textbf{Frasi relative a Servizio}}\\
        \hline
        Il marina offre una serie di servizi utili ai naviganti\footnote{Ad esempio lavanderie a gettoni, ristoranti, cantieri} . Ogni servizio è caratterizzato da un nome unico per ogni marina e gli orari di apertura e chiusura.\\
        
        Ogni servizio è amministrato da un \textbf{addetto} \\
        \hline
    \end{tabularx}
\end{center}

\begin{center}
    \begin{tabularx}{\textwidth}{|X|}
        \hline
        \rowcolor{gray!30}
        \multicolumn{1}{|c|}{\textbf{Frasi relative a Addetto}}\\
        \hline
        un \textbf{addetto} del quale ci interessa la data di inizio contratto e la fine. Se la fine non è segnata allora lavorerà li a tempo indeterminato.\\
        \hline
    \end{tabularx}
\end{center}

\begin{center}
    \begin{tabularx}{\textwidth}{|X|}
        \hline
        \rowcolor{gray!30}
        \multicolumn{1}{|c|}{\textbf{Frasi relative a Allacciamento}}\\
        \hline
        Gli allacciamenti hanno un nome, un'unità di misura ed un costo unitario.\\
        Relativamente agli allacciamenti ne vengono anche registrati i consumi da parte di ogni cliente\\
        \hline
    \end{tabularx}
\end{center}


\begin{center}
    \begin{tabularx}{\textwidth}{|X|}
        \hline
        \rowcolor{gray!30}
        \multicolumn{1}{|c|}{\textbf{Frasi relative a Corso}}\\
        \hline
        Questi corsi utilizzano a volte le imbarcazioni messe a disposizione da alcune persone e ne viene registrato il nome, il prezzo per la partecipazione, la data di inizio e la data di fine del corso. Esisteranno quindi dei clienti senza imbarcazione ma che partecipano ad uno dei corsi offerti dal marina.\\
        \hline
    \end{tabularx}
\end{center}

\begin{center}
    \begin{tabularx}{\textwidth}{|X|}
        \hline
        \rowcolor{gray!30}
        \multicolumn{1}{|c|}{\textbf{Frasi relative a Fattura}}\\
        \hline
        la quantità da pagare mensilmente viene quindi fatturata al cliente e viene registrato se la fattura è stata pagata oppure no.\\
        \hline
    \end{tabularx}
\end{center}

\begin{center}
    \begin{tabularx}{\textwidth}{|X|}
        \hline
        \rowcolor{gray!30}
        \multicolumn{1}{|c|}{\textbf{Frasi relative a Consumo}}\\
        \hline
        vengono anche registrati i consumi da parte di ogni cliente, come procedura interna del marina viene fatta una lettura del contatore all'arrivo dell'imbarcazione ed una lettura del contatore alla partenza della stessa. Se l'imbarcazione non ha una data di prevista partenza, ad esempio un cliente che ormeggia la sua imbarcazione a tempo indefinito la lettura viene fatta mensilmente e registrata.\\
        \hline
    \end{tabularx}
\end{center}

\subsection{Operazioni tipiche}
Contando che ad un marina appartengono dai 50 ai 500 moli e una delle aziende o associazioni che utilizzerà l'applicativo potrebbe possedere da 1 a 5 marina in media le operazioni effettuate nel marina con annessa frequenza sono le seguenti.
\begin{center}
    \begin{tabularx}{\textwidth}{|p{90mm}|X|}
        \hline
        \rowcolor{gray!30}
        \textbf{Operazione} & \textbf{Frequenza}\\
        \hline
        1. Stampa della fattura di pagamento del cliente & 40 volte al giorno\\
        \hline
        2. Controllo dei posti disponibili per una certa imbarcazione(Controllo delle dimensioni)& 40 volte al giorno\\
        \hline
        3. Organizzazione di un corso & 30 volte l'anno\\
        \hline
        4. Cambio dell'addetto ad un servizio & 1-2 volte ogni due anni\\
        \hline
        5. Prenotazioni di posti barca & 30-50 volte al giorno\\
        \hline
        6. Arrivo di un'imbarcazione in uno dei marina & 30-40 volte al giorno\\
        \hline
        7. Espletamento degli obblighi doganali & 140 volte a settimana\\
        \hline
    \end{tabularx}
\end{center}
