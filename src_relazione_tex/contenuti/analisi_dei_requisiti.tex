
\section{Analisi dei requisiti}
\subsection{Descrizione testuale}
Si vuole realizzare una base di dati per l'applicazione \textit{Gestionale servizi del diporto} che dovrà gestire tutti i dati relativi ad un marina e le \textbf{imbarcazioni} che vi sosteranno, i \textbf{servizi} offerti e i dati relativi ai \textbf{clienti}.

Le fuznzioni desiderate prevedono: il calcolo delle \textbf{fatture} dei clienti; la gestione di \textbf{soste} e \textbf{prenotazioni} con recupero dei posti disponibili; la presentazione dei \textbf{servizi} aperti ai clienti.

Il marina in esame ha circa 250 \textbf{posti barca}, numero che cambia solo in caso di ristrutturazioni particolari o durante ampliamenti del marina. Il traffico di \textbf{imbarcazioni}, \textbf{persone} ed informazioni è quindi commisurato a questo numero di \textbf{posti barca}.

Alcuni dei \textbf{clienti} del marina sono \textit{occasionali}, ad esempio persone che effettuano crociere e navigano di porto in porto; Altri clienti sono \textit{abituali} e tengono ormeggiate le loro imbarcazioni nel marina per molti mesi all'anno se non addirittura a tempo indeterminato. Dei primi si vuole sapere quante volte hanno ormeggiato nel marina, dei secondi si vuole tenere nota del loro sconto personale applicato alle fatture. Spesso un \textit{cliente occasionale} diventa un \textit{cliente abituale} e quando questo accade gli viene assegnato uno sconto deciso per persona.

Dei \textbf{moli}\footnote{Sinonimo di posto barca} si vuole tenere traccia della loro occupazione nel tempo per fini statistici\footnote{Ad esempio il \textbf{moli} preferito da una determinata persona o il \textbf{moli} più utilizzato}. Per ogni \textbf{molo}, identificato da un numero positivo, è importante memorizzare le dimensioni(Molto importanti, poiché solo le imbarcazioni più piccole di queste dimensioni potranno ormeggiare in questo \textbf{moli}),il prezzo giornaliero , gli allacciamenti\footnote{Si intendono allacciamenti idrici, elettrici, di aria compressa} e lo stato di occupazione.

Gli \textbf{allacciamenti} sono identificati dal nome e hanno un'unità di misura ed un costo unitario.

Relativamente agli \textbf{allacciamenti} ne vengono anche registrati i \textbf{consumi} da parte di ogni cliente, come procedura interna del marina viene fatta una lettura del contatore ogni inizio del mese\footnote{Chiaramente solo se c'è una barca in sosta}. Se l'imbarcazione arriva e/o parte in momenti differenti dall'inizio del mese la lettura viene fatta in quelle due occasioni.

Questi \textbf{consumi}, assieme alle \textbf{soste} effettuate, vengono poi fatturati al cliente e viene registrato se la \textbf{fattura} è stata pagata oppure no.
La \textbf{fattura} è identificata univocamente dal cliente destinatario e la data di scadenza della stessa.

Da quando l'applicativo verrà messo in funzione si inizieranno a memorizzare le \textbf{soste}.

Le \textbf{imbarcazioni} che sosteranno nel marina sosteranno nel tempo in più moli differenti, e chiaramente i \textbf{moli} accoglieranno nel tempo imbarcazioni differenti, ma non contemporaneamente. Un \textbf{molo} può infatti accogliere solo un' \textbf{imbarcazione} alla volta.

Di ogni \textbf{imbarcazione} vengono registrati codice internazionale univoco, la bandiera battuta, il nome del capitano, le dimensioni\footnote{Definite come: Pescaggio,Larghezza e lunghezza fuoritutto con acronimo inglese LOA}, il numero di posti letto e se è presente anche il nome. Queste informazioni vengono memorizzate assieme ai dati dei \textbf{clienti}, ovvero i proprietari delle suddette \textbf{imbarcazioni} per onorare gli obblighi di registrazione doganale.

I \textbf{clienti} sono coloro che possiedono un'\textbf{imbarcazione} ormeggiata all'interno del marina e di loro ne viene salvato il nome, il cognome, il codice fiscale, gli eventuali contatti, la residenza, la cittadinanza, la data di nascita.

Ogni \textbf{cliente} può quindi prenotare un \textbf{molo} del marina preventivamente. Ogni \textbf{prenotazione} è caratterizzata da una data prevista di arrivo, una data prevista di partenza, il \textbf{molo} interessato, il cliente prenotante e un informazione per capire se la prenotazione è stata trasformata in sosta oppure no.

Per controllare se un \textbf{molo} è disponibile bisogna quindi controllare se il \textbf{molo} è al momento occupato e inoltre se esistono già delle \textbf{prenotazioni} per le date in cui si prevede che l'\textbf{imbarcazione} sosti. Se la \textbf{sosta} prevista è a tempo indeterminato ma una prenotazione parte più tardi,l'imbarcazione può essere spostata a sostare in un altro \textbf{moli} non prenotato.

Il marina offre poi una serie di \textbf{servizi} utili ai naviganti\footnote{Ad esempio lavanderie a gettoni, ristoranti, cantieri} . Ogni \textbf{servizio} è caratterizzato da un nome unico e gli \textbf{orari} in cui è aperto nei vari giorni di una settimana.

Ogni servizio è amministrato da un \textbf{addetto} del quale ci interessa la data di inizio contratto e la fine. Se la fine non è segnata allora lavorerà li a tempo indeterminato. Ogni \textbf{addetto} è definito univocamente dalla data di inizio contratto e il servizio che gestisce. per questioni legate al loro contratto di lavoro ci interessa di loro anche il codice fiscale e la data di nascita, oltre al loro nome, cognome e ai loro contatti.

Gli \textbf{addetti} inoltre spesso sono anche \textbf{clienti}. Una \textbf{persona} può essere quindi o un \textbf{addetto} o un \textbf{cliente} o entrambi.

\subsection{Glossario dei termini}

\begin{center}
    \begin{tabularx}{\textwidth}{|>{\RaggedRight}p{2cm}|p{7cm}|p{2.4cm}|>{\RaggedRight}X|}
        \hline
        \rowcolor{gray!30}
        \textbf{Entità} & \textbf{Descrizione} & \textbf{Sinonimi} & \textbf{collegamenti} \\
        \hline
        Imbarcazione & Il mezzo di trasporto ormeggiato nei moli &  & Cliente, Sosta\\
        
        \hline
        Molo & Il luogo in cui le imbarcazioni sostano & Posto barca, ormeggio & Sosta, Prenotazione, Allacciamento\\
        
        \hline
        Servizio & Attività di varia utilità rivolte ai clienti del marina &  & Periodo apertura, addetto \\
        
        \hline
        Addetto & Persona che gestisce un servizio & Responsabile & Servizio, Persona \\

        \hline
        Cliente& Persona che possiede un'imbarcazione ormeggiata nel marina & Armatore & Entità figlia di Persona, Imbarcazione, Consumo, Prenotazione\\
        
        \hline
        Cliente occasionale &Persone che viaggiano salturiamente & & Entità figlia di Cliente  \\
         
        \hline
        Cliente abituale & Persone che utilizzano stabilmente un marina &  & Entità figlia di Cliente \\
          
        \hline
        Persona& Tutte le persone che frequentano un marina   &  & Padre di Cliente e Addetto \\  
        
        \hline
        Prenotazione & Tutte le prenotazioni dei clienti &  & Molo, cliente \\
        
        \hline
        Sosta & Possibilità data ai clienti delle imbarcazioni di un marina  & & Imbarcazione, Molo \\
        
        \hline
        Allacciamento & Un servizio consumabile tipo acqua o elettricità disponibile ad un molo & Fornitura & Molo, Consumo\\
        
        \hline
        Periodo di apertura & Orari in cui un servizio è aperto durante la settimana & & Servizio\\

        \hline
        Consumo & Utilizzo misurato di un allacciamento presente ai moli & & Allacciamento, Cliente, Fattura \\
       
        \hline
        Fattura & Ricevuta sui consumi e le soste effettuate da parte dei clienti & Ricevuta, scontrino & Cliente, Consumo, Sosta\\
        
        \hline
    \end{tabularx}
\end{center}

\subsection{Strutturazione dei requisiti}

\begin{center}
    \begin{tabularx}{\textwidth}{|X|}
        \hline
        \rowcolor{gray!30}
        \multicolumn{1}{|c|}{\textbf{Frasi relative a Imbarcazione}}\\
        \hline
        Il traffico di \textbf{imbarcazioni}, \textbf{persone} ed informazioni è quindi commisurato a questo numero di \textbf{posti barca}. \\

        Le \textbf{imbarcazioni} che sosteranno nel marina sosteranno nel tempo in più \textbf{moli} differenti, e chiaramente i \textbf{moli} accoglieranno nel tempo \textbf{imbarcazioni} differenti, ma non contemporaneamente. Un \textbf{moli} può infatti accogliere solo un’ imbarcazione alla volta. \\

        Di ogni imbarcazione vengono registrati codice internazionale univoco, la bandiera battuta, il nome del cap- itano, le dimensioni5, il numero di posti letto e se è presente anche il nome.\\
        \hline
    \end{tabularx}
\end{center}

\begin{center}
    \begin{tabularx}{\textwidth}{|X|}
        \hline
        \rowcolor{gray!30}
        \multicolumn{1}{|c|}{\textbf{Frasi relative a Molo}}\\
        \hline
        Il marina in esame ha circa 250 posti barca, numero che cambia solo in caso di ristrutturazioni particolari o durante ampliamenti del marina. \\

        Dei \textbf{moli} si vuole tenere traccia della loro occupazione nel tempo per fini statistici.\\
        
        Per ogni \textbf{moli}, identificato da un numero positivo, è importante memorizzare le dimensioni ,il prezzo giornaliero , gli allacciamenti e lo stato di occupazione.\\

        Per controllare se un \textbf{moli} è disponibile bisogna quindi controllare se il \textbf{moli} è al momento occupato e inoltre se esistono già delle prenotazioni per le date in cui si prevede che l’imbarcazione sosti. Se la sosta prevista è a tempo indeterminato ma una prenotazione parte più tardi,l’imbarcazione può essere spostata a sostare in un altro \textbf{moli} non prenotato.\\

        \hline
    \end{tabularx}
\end{center}

\begin{center}
    \begin{tabularx}{\textwidth}{|X|}
        \hline
        \rowcolor{gray!30}
        \multicolumn{1}{|c|}{\textbf{Frasi relative a Servizio}}\\
        \hline
        Il marina offre poi una serie di servizi utili ai naviganti . Ogni servizio è caratterizzato da un nome unico e gli orari in cui è aperto nei vari giorni di una settimana.
Ogni servizio è amministrato da un addetto  \\
        \hline
    \end{tabularx}
\end{center}

\begin{center}
    \begin{tabularx}{\textwidth}{|X|}
        \hline
        \rowcolor{gray!30}
        \multicolumn{1}{|c|}{\textbf{Frasi relative a Addetto}}\\
        \hline
        un addetto del quale ci interessa la data di inizio contratto e la fine. Se la fine non è segnata allora lavorerà li a tempo indeterminato. Ogni addetto è definito univocamente dalla data di inizio contratto e il servizio che gestisce. per questioni legate al loro contratto di lavoro ci interessa di loro anche il codice fiscale e la data di nascita, oltre al loro nome, cognome e ai loro contatti.\\
        
        Gli addetti inoltre spesso sono anche clienti. \\
        \hline
    \end{tabularx}
\end{center}

\begin{center}
    \begin{tabularx}{\textwidth}{|X|}
        \hline
        \rowcolor{gray!30}
        \multicolumn{1}{|c|}{\textbf{Frasi relative a Cliente}}\\
        \hline
        ... assieme ai dati dei clienti, ovvero i proprietari delle suddette \textbf{imbarcazioni} per onorare gli obbllighi di registrazione doganale.\\
        
        I clienti sono coloro che possiedono un’imbarcazione ormeggiata all’interno del marina e di loro ne viene salvato il nome, il cognome, il codice fiscale, gli eventuali contatti, la residenza, la cittadinanza, la data di nascita.\\

        Ogni cliente può quindi prenotare un \textbf{moli} del marina preventivamente.\\
        \hline
    \end{tabularx}
\end{center}

\begin{center}
    \begin{tabularx}{\textwidth}{|X|}
        \hline
        \rowcolor{gray!30}
        \multicolumn{1}{|c|}{\textbf{Frasi relative a Cliente occasionale}}\\
        \hline
        Alcuni dei clienti del marina sono occasionali, ad esempio persone che effettuano crociere e navigano di porto in porto; \\

        Dei primi si vuole sapere quante volte hanno ormeggiato nel marina, \\

        Spesso un cliente occasionale diventa un cliente abituale e quando questo accade gli viene assegnato uno sconto deciso per persona.\\


        \hline
    \end{tabularx}
\end{center}

\begin{center}
    \begin{tabularx}{\textwidth}{|X|}
        \hline
        \rowcolor{gray!30}
        \multicolumn{1}{|c|}{\textbf{Frasi relative a Cliente abituale}}\\
        \hline
        Altri clienti sono abituali e tengono ormeggiate le loro \textbf{imbarcazioni} nel marina per molti mesi all’anno se non addirittura a tempo indeterminato. \\

        dei secondi si vuole tenere nota del loro sconto personale applicato alle fatture. \\
        \hline
    \end{tabularx}
\end{center}

\begin{center}
    \begin{tabularx}{\textwidth}{|X|}
        \hline
        \rowcolor{gray!30}
        \multicolumn{1}{|c|}{\textbf{Frasi relative a Persona}}\\
        \hline
        Una \textbf{persona} può essere quindi o un \textbf{addetto} o un \textbf{cliente} o entrambi. \\

        \hline
    \end{tabularx}
\end{center}

\begin{center}
    \begin{tabularx}{\textwidth}{|X|}
        \hline
        \rowcolor{gray!30}
        \multicolumn{1}{|c|}{\textbf{Frasi relative a Prenotazione}}\\
        \hline
        Ogni \textbf{prenotazione} è caratterizzata da una data prevista di arrivo, una data prevista di partenza, il \textbf{molo} interessato, il cliente prenotante e un informazione per capire se la prenotazione è stata trasformata in sosta oppure no.\\

        \hline
    \end{tabularx}
\end{center}



\begin{center}
    \begin{tabularx}{\textwidth}{|X|}
        \hline
        \rowcolor{gray!30}
        \multicolumn{1}{|c|}{\textbf{Frasi relative a sosta}}\\
        \hline
        Da quando l’applicativo verrà messo in funzione si inizieranno a memorizzare le soste.\\

        \hline
    \end{tabularx}
\end{center}

\begin{center}
    \begin{tabularx}{\textwidth}{|X|}
        \hline
        \rowcolor{gray!30}
        \multicolumn{1}{|c|}{\textbf{Frasi relative a Allacciamento}}\\
        \hline
        Gli allacciamenti sono identificati dal nome e hanno un’unità di misura ed un costo unitario. \\


        \hline
    \end{tabularx}
\end{center}

\begin{center}
    \begin{tabularx}{\textwidth}{|X|}
        \hline
        \rowcolor{gray!30}
        \multicolumn{1}{|c|}{\textbf{Frasi relative a Periodo di apertura}}\\
        \hline
        Ogni servizio è caratterizzato da un nome unico e gli orari in cui è aperto nei vari giorni di una settimana. \\
        \hline
    \end{tabularx}
\end{center}

\begin{center}
    \begin{tabularx}{\textwidth}{|X|}
        \hline
        \rowcolor{gray!30}
        \multicolumn{1}{|c|}{\textbf{Frasi relative a Consumo}}\\
        \hline
        i consumi da parte di ogni cliente, come proce- dura interna del marina viene fatta una lettura del contatore ogni inizio del mese4. Se l’imbarcazione arriva e/o parte in momenti differenti dall’inizio del mese la lettura viene fatta in quelle due occasioni. \\

        \hline
    \end{tabularx}
\end{center}

\begin{center}
    \begin{tabularx}{\textwidth}{|X|}
        \hline
        \rowcolor{gray!30}
        \multicolumn{1}{|c|}{\textbf{Frasi relative a Fattura}}\\
        \hline
        Questi consumi, assieme alle soste effettuate, vengono poi fatturati al cliente e viene registrato se la fattura è stata pagata oppure no. La fattura è identificata univocamente dal cliente destinatario e la data di scadenza della stessa.\\

        \hline
    \end{tabularx}
\end{center}

\subsection{Operazioni tipiche}
Contando che al marina appartengono circa 250 \textbf{moli}, in media le operazioni effettuate nel marina con annessa frequenza sono le seguenti.
\begin{center}
    \begin{tabularx}{\textwidth}{|p{90mm}|X|}
        \hline
        \rowcolor{gray!30}
        \textbf{Operazione} & \textbf{Frequenza}\\
        \hline
        1. Stampa della fattura di pagamento di un cliente & 40 volte al giorno\\

        \hline
        2. Generazione conteggi fatture clienti & 30-40 volte al giorno più picco mensile a fine mese\footnote{Vengono calcolate le fatture mensili di tutti i clienti che stanno oltre il mese}\\

        \hline
        3. Controllo dei posti disponibili per una certa imbarcazione(Controllo delle dimensioni)& 40 volte al giorno\\

        \hline
        4. Cambio dell'addetto ad un servizio & 1-2 volte ogni due anni\\

        \hline
        5. Prenotazioni di posti barca & 30-50 volte al giorno\\

        \hline
        6. Arrivo di un'imbarcazione nel marina & 30-40 volte al giorno\\

        \hline
        7. Consultazione delle aperture di un servizio & 20-40 volte al giorno\\

        \hline
    \end{tabularx}
\end{center}
