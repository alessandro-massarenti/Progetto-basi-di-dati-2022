\section{Progettazione logica}

\subsection{Analisi delle ridondanze}
Nello schema concettuale possiamo individuare stato di occupazione che è attributo di molo e deducibile dalla relazione sosta, la quale presenta l'attributo data di arrivo e data di partenza.

Le operazioni coinvolte sono la numero 3 e la numero 6 che presentano una frequenza di 40 volte al giorno e 30-40 volte ogni giorno.

Si procede alla valutazione del costo totale in termini di accessi nel caso di ridondanza e di assenza di essa.

\begin{center}
    \begin{tabularx}{\textwidth}{|p{90mm}|X|}
        \hline
        \rowcolor{gray!30}
        \textbf{Operazione} & \textbf{Frequenza}\\
        \hline
        3. Controllo dei posti disponibili per una certa imbarcazione(Controllo delle dimensioni)& 40 volte al giorno\\
        \hline
        6. Arrivo di un'imbarcazione nel marina & 30-40 volte al giorno\\
        \hline
    \end{tabularx}
\end{center}

Ci riferiremo per i nostri calcoli a dei volumi plausibili nella vita della base di dati. I moli non variano particolarmente di numero, le imbarcazioni vengono registrate al loro primo accesso e le soste aumentano ogni giorno. Mediamente un'imbarcazione sosta 2,7 volte in un molo.

\begin{center}
    \begin{tabularx}{\textwidth}{|X|X|X|}
        \hline
        \rowcolor{gray!30}
        \textbf{Concetto} & \textbf{Tipo} & \textbf{Volume}\\
        \hline
        Molo & ENTITÀ & 250\\
        \hline
        Imbarcazione & ENTITÀ & 5000\\
        \hline
        Sosta & ENTITÀ & 13500\\ % Contando che ogni imbarcazione sosta in media due volte per molo.
        \hline
        M\_S & RELAZIONE & 13500\\ % Contando che ogni imbarcazione sosta in media due volte per molo.
        \hline
    \end{tabularx}
\end{center}

\paragraph{operazione 3}

\begin{center}
    \begin{minipage}{.48\linewidth}
        \begin{tabularx}{\linewidth}{|X|l|l|l|}
            \hline
            \rowcolor{gray!30}
            \multicolumn{4}{|c|}{\textbf{Con ridondanza}}\\
            \hline
            \rowcolor{gray!15}
            Concetto & Costrutto & Accessi & Tipo\\
            \hline
            Molo & E & 250 & L\\
            \hline
        \end{tabularx}
    \end{minipage}
    \begin{minipage}{.48\linewidth}
        \begin{tabularx}{\linewidth}{|X|l|l|l|}
            \hline
            \rowcolor{gray!30}
            \multicolumn{4}{|c|}{\textbf{Senza ridondanza}}\\
            \hline
            \rowcolor{gray!15}
            Concetto & Costrutto & Accessi & Tipo\\
            \hline
            Sosta & E & 13500 & L\\
            \hline
            M\_S & R & 13500 & L\\
            \hline
            Molo & E & 250 & L\\
            \hline
        \end{tabularx}
    \end{minipage}
\end{center}

\underline{Con ridondanza}
\begin{itemize}
    \item Totale accessi(Solo in lettura): $250\xrightarrow{*40} 10000$
\end{itemize}

\underline{Senza ridondanza}
\begin{itemize}
    \item Totale accessi(Solo in lettura): $250+13500+13500 = 27250 \xrightarrow{*40} 1090000$ Giornalieri
\end{itemize}

\paragraph{operazione 6}

Assumo che l'imbarcazione sia già presente perché l'esserci o non esserci non cambia il calcolo della ridondanza.

\begin{center}
\begin{tabularx}{\linewidth}{|X|}
    \hline
    \rowcolor{gray!30}
    \multicolumn{1}{|c|}{\textbf{Operazioni principali}}\\
    \hline
    Cerco l'imbarcazione\\
    \hline
    Cerco i moli disponibili\\
    \hline
    Confronto le dimensioni dell'imbarcazione con i moli disponibili\\
    \hline
    Scrivo la sosta\\
    \hline
\end{tabularx}
\end{center}


\begin{center}
    \begin{minipage}{.48\linewidth}
        \begin{tabularx}{\linewidth}{|X|l|l|l|}
            \hline
            \rowcolor{gray!30}
            \multicolumn{4}{|c|}{\textbf{Con ridondanza}}\\
            \hline
            \rowcolor{gray!15}
            Concetto & Costrutto & Accessi & Tipo\\
            \hline
            Imbarcazione & E & 1 & L\\
            \hline
            Molo & E & 250 & L\\
            \hline
            Sosta & E & 1 & S\\
            \hline
            Molo & E & 1 & S\\
            \hline
        \end{tabularx}
    \end{minipage}
    \begin{minipage}{.48\linewidth}
        \begin{tabularx}{\linewidth}{|X|l|l|l|}
            \hline
            \rowcolor{gray!30}
            \multicolumn{4}{|c|}{\textbf{Senza ridondanza}}\\
            \hline
            \rowcolor{gray!15}
            Concetto & Costrutto & Accessi & Tipo\\
            \hline
            Imbarcazione & E & 1 & L\\
            \hline
            Sosta & E & 13500 & L\\
            \hline
            M\_S & R & 13500 & L\\
            \hline
            Molo & E & 250 & L\\
            \hline
            Sosta & E & 1 & S\\
            \hline
        \end{tabularx}
    \end{minipage}
\end{center}


\underline{Con ridondanza}
\begin{itemize}
    \item Totale scritture: $1 + 1 = 2 \xrightarrow{*2} 4$ 
    \item Totale letture: $250 + 1 = 251$
    \item Totale accessi: $4+251 = 255 \xrightarrow{*40} 102000$ Giornalieri
\end{itemize}
\underline{Senza ridondanza}
\begin{itemize}
    \item Totale scritture: $1 \xrightarrow{*2} 2$ 
    \item Totale letture: $1 + 13500 + 13500 + 250 = 27251$
    \item Totale accessi: $2+27251 = 27253 \xrightarrow{*40} 1090120$ Giornalieri
\end{itemize}

\paragraph{In conclusione}
Nell'\textbf{operazione 2} e nell'\textbf{operazione 6} risulta meno costosa in termini di accessi l'operazione con ridondanza. Notando un grosso divario tra le operazioni con ridondanza e le operazioni senza ridondanza, qui si sceglie di mantenere la ridondanza diminuendo il numero di accessi necessari.

\subsubsection{altre ridondanze}

Un altra ridondanza è Quantità di soste in \textbf{Cliente occasionale} è deducibile da Cliente possiede imbarcazione che sosta in un molo andando a vedere quante soste hanno effettuato le sue imbarcazioni;

In seguito all'analisi sugli accessi abbiamo stimato che risulti essere preferibile mantenere la ridondanza relativamente ad eventuali operazioni di lettura e scrittura di questi valori, 
pertanto si è deciso di mantenere la ridondanza.


\subsection{Eliminazione degli attributi multi-valore}

Contatto in Persona è multi-valore.

Abbiamo quindi reificato contatto in una relazione binaria come segue.

\begin{center}
    \includegraphics[width=\linewidth / 2]{img/multi_valore.png}
\end{center}


\subsection{Eliminazione delle generalizzazioni}

\paragraph{Cliente}\mbox{}\\
Si accorpano Cliente abituale e cliente occasionale a Cliente. Dato che la quantità di soste è un attributo utile per entrambe le tipologie di cliente. Lo sconto personale sarà invece impostato ad un valore che indica la percentuale di sconto se il cliente è abituale e a NULL se il cliente è occasionale.

\paragraph{Persona}\mbox{}\\
Nelle generalizzazioni di persona, per evitare inutili valori NULL in un eventuale accorpamento \textbf{Persona} viene mantenuta come entità. Questa entità conterrà tutti i dati relativi alla persona e avrà due entità deboli collegate ovvero \textbf{Addetto} e \textbf{Cliente}.

Persona avrà da 0 a 1 \textbf{Cliente} e da 0 a 1 \textbf{Addetto}. Chiaramente ogni \textbf{Cliente}  e \textbf{Addetto} avrà una e una sola persona legata. In \textbf{Addetto} e \textbf{Cliente} posizioneremo i dati relativi a queste due entità come da schema concettuale.

\subsection{Modifiche, aggiunte e chiarimenti alle chiavi}

Tutte le chiavi primarie sono definite utilizzando le chiavi della progettazione concettuale, tranne nei seguenti casi.

\paragraph{Imbarcazione} Ad \textbf{Imbarcazione} oltre alla sua chiave primaria Codice internazionale viene aggiunto un id autoincrementante utilizzato come chiave\footnote{Chiave unica}. Questo è stato scelto per poter utilizzare il costraint relativo alla sovrapposizione delle soste. Infatti esso funziona solo con attributi interi e non con Varchar\footnote{A meno di installare il pacchetto aggiuntivo di Postgres che lo permetta ma per lo scopo di questo progetto abbiamo preferito evitare per questioni di compatibilità.}.

\paragraph{Cliente} A \textbf{Cliente} oltre alla sua chiave primaria \textit{Codice fiscale} viene aggiunto un id autoincrementante utilizzato come chiave\footnote{Chiave unica}. Questo è stato scelto per poter utilizzare il costraint relativo alla sovrapposizione delle prenotazioni. Infatti esso funziona solo con attributi interi e non con Varchar.

\paragraph{Addetto} Addetto concettualmente eredita la chiave primaria di \textbf{Persona} ma come da progettazione concettuale mantiene la chiave su \underline{Servizio gestito, Data inizio contratto}. Come chiave primaria si sceglie quindi di utilizzare il \textit{codice fiscale} derivante da \textbf{persona} perché più leggero.

\paragraph{Periodo di apertura} Per ridurre lo spazio utilizzato nella tabella che rappresenta la relazione \textit{Apertura}, a \textbf{Periodo di apertura} abbiamo aggiunto una chiave primaria tramite id autoincrementante. Chiaramente non abbiamo rimosso la chiave unica già presente, infatti essa garantisce l'unicità delle triple presenti.

\paragraph{Sosta} Per semplificare il lavoro con le prenotazioni trasformate in \textbf{Sosta} a sosta viene aggiunto un id autoincrementante come chiave. Questo id verrà utilizzato da \textbf{Prenotazione} per definire se la prenotazione è stata trasformata in sosta.


\subsection{Schema concettuale ristrutturato - Schema logico}
\includegraphics[width=\textwidth]{img/erlogico.png}


\subsection{Descrizione schema relazionale}

Per questione di compatibilità con il \textit{DBMS} alcuni nomi di attributi entità e relazioni sono stati normalizzati, utilizzando il camelCase, togliendo gli accenti, accorciando i nomi molto lunghi e con altre piccole accortezze.\\

\textbf{Imbarcazione}(\underline{\textit{MMSI}}, \underline{id}, cliente, bandiera, nomeCapitano, nPostiLetto, nome, pescaggio, larghezza, LOA);

\textbf{Molo} (\underline{\textit{id}}, occupato, profonditaMinima, larghezza, lunghezza, prezzoGiorno);

\textbf{Servizio} (\underline{\textit{nome}});

\textbf{Addetto} (\underline{\textit{persona}}, \underline{servizio, inizioContratto}, fineContratto);

\textbf{Cliente} (\underline{\textit{persona}}, \underline{id}, cittadinanza, residenza, quantitaSoste, scontoPersonale);

\textbf{Persona} (\underline{\textit{CF}}, dataNascita, nome, cognome);

\textbf{Prenotazione} (\underline{\textit{cliente, prevArrivo, molo}}, prevPartenza, sosta);

\textbf{Sosta} (\underline{\textit{imbarcazione, molo, arrivo}}, \underline{id}, partenza, fattura);

\textbf{Allacciamento} (\underline{\textit{nome}}, prezzoUnitario, unitaMisura);

\textbf{Fornitura} (\underline{\textit{allacciamento, molo}});

\textbf{PeriodoApertura} (\underline{\textit{id}}, \underline{giorno, apertura, chiusura});

\textbf{AperturaServizio} (\underline{\textit{servizio, periodoApertura}});

\textbf{Consumo}(\underline{\textit{cliente, allaciamento, inizio}}, fine, quantita, fattura);

\textbf{Fattura}(\underline{\textit{id}},\underline{cliente, scadenza}, pagato);\\

La chiave primaria è la prima delle chiavi indicate dalla sottilineatura.

\subsection{Vincoli di integrità referenziali}

\textbf{Imbarcazione}.cliente -> \textit{Cliente}.persona\\
\textbf{Addetto}.persona -> \textit{Persona}.CF\\
\textbf{Cliente}.persona -> \textit{Persona}.CF\\
\textbf{Prenotazione}.cliente -> \textit{Cliente}.id\\
\textbf{Prenotazione}.molo -> \textit{Molo}.id\\
\textbf{Prenotazione}.sosta -> \textit{Sosta}.id\\
\textbf{Sosta}.imbarcazione -> \textit{Imbarcazione}.id\\
\textbf{Sosta}.molo -> \textit{Molo}.id\\
\textbf{Sosta}.fattura -> \textit{Fattura}.id\\
\textbf{Fornitura}.allacciamento -> \textit{Allacciamento}.nome\\
\textbf{Fornitura}.molo -> \textit{Molo}.id\\
\textbf{AperturaServizio}.servizio -> \textit{Servizio}.nome\\
\textbf{AperturaServizio}.periodoApertura -> \textit{PeriodoApertura}.id\\
\textbf{Consumo}.cliente -> \textit{Cliente}.persona\\
\textbf{Consumo}.allacciamento -> \textit{Allacciamento}.nome\\
\textbf{Consumo}.fattura -> \textit{Fattura}.id\\
\textbf{Fattura}.cliente -> \textit{Cliente}.persona\\

\subsection{Check e costraint}

\subsubsection{Vincoli non gestiti}
Non è stato purtroppo possibile mantenere fede a tutti i vincoli posti nella progettazione concettuale, e relativamente ad essi viene lasciato al programmatore che utilizzerà la base dati l'onere di mantenerla consistente.

\paragraph{Fattura}
Non viene gestito il vincolo relativo a fattura, per via di una dipendenza circolare tra la generazione della fattura e la creazione delle sue righe. Si considera quindi la fattura con zero linee come "\textit{fattura da riempire}".

\paragraph{Molo}
Per gestire la consistenza di \textbf{Molo} si possono utilizzare dei trigger che la controllino ad ogni variazione dei dati. Ma per ridurre la complessità si è deciso anche qui di lasciare l'onere della consistenza al programmatore. In caso di riscontrate inconsistenze sarà comunque sempre possibile fare affidamento sulle soste le quali saranno sempre consistenti.

\paragraph{Sosta}
Un altro vincolo purtroppo non gestito è la palese impossibilità di inserire un'imbarcazione in un molo più piccolo di essa. Questo vincolo non è gestibile, se non con soluzioni inutilmente complesse, in quanto al momento Postgres non permette di fare controlli su altre tabelle al momento dell'inserimento di un dato. Si può però comunque notare che questo vincolo sarà rispettato fisicamente nella realtà.

\paragraph{Prenotazione} In prenotazione, per lo stesso motivo posto al paragrafo superiore non sarà possibile controllare nelle soste o nei moli se al moemento della prenotazione sono già occupati. Questo vincolo è però in buona parte rispettato dato che nelle soste si inseriscono solo date attuali o postume e non date relative al futuro, mentre in Prenotazioni si inseriscono date future. Si nota quindi come i problemi affrontati dalle due entità non collidano in molti posti, e sarà in buona parte necessario controllare solo in Prenotazione. Si lascia in ogni caso al programmatore l'onere di mantenere il vincolo.

\subsubsection{Vincoli gestiti}\label{vincoli_gestiti}

\paragraph{PeriodoApertura}
In \textbf{PeriodoApertura} è attivato un check che controlla che l'orario di chiusura sia sempre postumo all'orario di apertura.

\paragraph{Prenotazione e Sosta}
In \textbf{Prenotazione} ed in \textbf{Sosta} viene eseguito un check simile a quello di \textbf{PeriodoApertura}. In queste due entità viene quindi controllato che il momento della partenza sia sempre postumo al momento dell'arrivo, questo anche quando la partenza è non definita. Per fare questo si è quindi deciso di settare la partenza con valore di default ad "\textit{infinito}" in sosta.

In queste due entità vengono inoltre fatti due controlli relativi alla sovrapposizione di soste e prenotazioni.

Se un \textbf{Molo} è già occupato in un determinato periodo esso non sarà occupabile, e se un'\textbf{Imbarcazione} si trova già in un \textbf{Molo} non potrà sdoppiarsi ed essere sostante in più moli contemporaneamente.

Come precedentemente accennato per implementare questo costraint sono state fatte delle modifiche alle chiavi di imbarcazione e di cliente, aggiungendo una chiave intera in modo da poter utilizzare i GIST\footnote{Generalized Search Tree, ovvero un albero di ricerca molto simile al B-Tree} nella versione standard di Postgres\footnote{DBMS scelto per il progetto presente}.

\paragraph{fattura e consumo}

In fattura e consumo viene implementato un semplice check per controllare che le date di inizio, di fine, di emissione e di pagamento siano coerenti relativamente alle loro entità
