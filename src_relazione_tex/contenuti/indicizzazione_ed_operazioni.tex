%\section{Casistiche di utilizzo della base di dati}

%-	Quando un utente va via dovrà quindi pagare il conto, che è la somma di tutti i servizi di cui ha usufruito
%-	Avere facilmente in vista tutte le persone all’interno del marina
%-	Conoscere rapidamente quanti posti disponibili relativamente a certe dimensioni ci sono
%-	Ottenere i dati necessari a compilare i documenti doganali e relativi al soggiorno in uno stato estero per gli utenti con nazionalità differente dal marina


\section{Query}

\subsection{Classifica clienti più spendenti}
\subsection{Entrate di ogni anno divise in ricevute e da ricevere}
\subsection{Quantità di moli di ogni marina}

\subsection{Moli disponibili in un certo periodo per determinate dimensioni di barca}

\begin{lstlisting}
SELECT idmolo,dimensioni,marina from "Molo"
--SELECT count(idmolo) from "Molo"
JOIN "Dimensioni" d on d.id_dimensioni = "Molo".dimensioni

WHERE idmolo not in (
SELECT molo FROM "sosta"
WHERE data_arrivo<='2022-01-20 22:16:37.000000'
AND '2022-12-26 8:16:37.000000' <= data_partenza)

AND idmolo not in (
SELECT molo FROM "Prenotazione"
WHERE previsione_arrivo<='2022-01-20 22:16:37.000000'
AND '2022-12-26 8:16:37.000000' <= previsione_partenza)

AND lunghezza >= 9 and larghezza>= 3 and pescaggio >= 3
AND marina = '{45.47186543638522000,12.44861260105955800}'
ORDER BY dimensioni ASC;
LIMIT 10;
\end{lstlisting}

\subsection{Tutte le soste fatte dall'imbarcazione "Alexandra"}

Tutte le soste effettuate dall'imbarcazione "Alexandra" in uno dei marina ed in che date è arrivata in quel marina

\begin{lstlisting}
SELECT Mar.nome, I.nome, sosta.data_arrivo  FROM sosta
JOIN "Imbarcazione"  I on sosta.imbarcazione = I.codice_internazionale
JOIN "Molo" M on M.idmolo = sosta.molo
JOIN "Marina" Mar on M.marina = Mar.coordinate_geografiche

WHERE I.nome = 'Alexandra';
\end{lstlisting}

\subsection{Clienti con barca norvegese che hanno sostato a Portoferraio nel 2022}

\begin{lstlisting}
SELECT DISTINCT P.nome,P.cognome FROM sosta
JOIN "Imbarcazione" I on I.codice_internazionale = sosta.imbarcazione
JOIN "Molo" M on M.idmolo = sosta.molo
JOIN "Marina" mar on M.marina = mar.coordinate_geografiche
JOIN "Cliente" C on I.proprietario = C.persona
JOIN "Persona" P on C.persona = P.cf
WHERE data_arrivo>= '2022-01-01 00:00:00.000000'
AND data_partenza <= '2022-12-31 00:00:00.000000'
AND I.bandiera = 'Norvegia'
AND mar.citta = 'Portoferraio';
\end{lstlisting}

\section{Indicizzazione}

Essendo modificata pochissime volte\footnote{Ovvero solo in caso di ristrutturazioni del marina} e letta spesso in moltissime operazioni, l'entità \textbf{Molo} è importante che venga indicizzata tramite B-Tree, in questo modo, leggerne gli attributi relativi alle dimensioni è reso decisamente più efficiente.

Sono stati in secondo luogo, come precedentemente descritto al punto \ref{vincoli_gestiti} , utilizzati dei search tree per utilizzare al meglio i costraint relativi alle date ed orari di \textbf{prenotazioni} e \textbf{soste}.
