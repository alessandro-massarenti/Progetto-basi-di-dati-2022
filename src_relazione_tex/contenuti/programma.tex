\section{Esecuzione programma C++}
  Verificare all'interno del sorgente cpp che i parametri di connessione
  al database corrispondano (sono quelli di default, quindi se si sono apportate
  modifiche sarà necessario cambiarli).
  \begin{lstlisting}[language=c++]
    //Parametri database
    #define PG_HOST "127.0.0.1" // Indirizzo Host
    #define PG_USER "postgres" // Nome utente
    #define PG_DB "progetto_amassare_fzontaro" //Nome del database
    #define PG_PASS "******" //Password
    #define PG_PORT "5432"
  \end{lstlisting}

  \noindent
  Per prima cosa è necessario compilare il programma utilizzando il comando:
  
  \begin{lstlisting}[language=bash]
  $ compilatore_cpp codice.cpp -L dependencies/lib -lpq -o codice
  \end{lstlisting}

  \noindent
  Per eseguirlo (su sistemi UNIX like), basta utilizzare il comando:
  \begin{lstlisting}[language=bash]
  $ ./query
  \end{lstlisting}

  \noindent
  All'avvio del programma, verrà mostrata all'utente la lista delle query, potrà digitare un numero da 1 a 5 per eseguirle. Se la query selezionata utilizza dei parametri, verranno richiesti in input.
  \newline
  Tutte le query sono contenute all'interno di un unico file cpp.
  